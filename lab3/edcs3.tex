\documentclass[a4paper, 12pt]{article}
\usepackage[utf8x]{inputenc}
\usepackage[english, russian]{babel}
\usepackage[left=25mm, top=25mm, right=25mm, bottom=25mm]{geometry}
\usepackage{cmap}
\usepackage{indentfirst}
\usepackage{tikz}
\usepackage{float}
\usepackage{amsmath, amsfonts, amssymb}
\usepackage{graphicx}
\usepackage{hyperref}
\usepackage{listings}
\usepackage{caption}
\usepackage{subcaption}
\usepackage{xcolor}
\usepackage{etoolbox}
\usepackage{titlesec}
\usepackage{array}
\pagestyle{plain}
\patchcmd{\tableofcontents}{\contentsname}{\centering\contentsname}{}{}
\titleformat{\section}[block]{\normalfont\large\bfseries\centering}{}{0pt}{}
\titleformat{\subsection}[block]{\normalfont\normalsize\bfseries\centering}{}{0pt}{}
\allowdisplaybreaks
\graphicspath{{src/images/}}
\usetikzlibrary{patterns}
\definecolor{LightGray}{gray}{0.95}
\definecolor{LightGray2}{gray}{0.7}
\hypersetup{
    colorlinks=true,
    linkcolor=blue,
    filecolor=magenta,
    urlcolor=cyan,
    pdftitle={contents setup},
    pdfpagemode=FullScreen,
}


\begin{document}
    \begin{titlepage}

        \begin{center}
        Федеральное государственное автономное образовательное учреждение высшего образования
        «Национальный Исследовательский Университет ИТМО»
        \vfill
        
        \includegraphics[width=0.3\textwidth]{itmo.png} % requires /src/images/itmo.png

        {\large\bf ЛАБОРАТОРНАЯ РАБОТА №3}\\
        {\large\bf ПРЕДМЕТ «ЭЛЕКТРОННЫЕ УСТРОЙСТВА СИСТЕМ УПРАВЛЕНИЯ»}\\
        {\large\bf ТЕМА «ОПЕРАЦИОННЫЙ УСИЛИТЕЛЬ В ОСНОВНЫХ СХЕМАХ ВКЛЮЧЕНИЯ»}\\
        Вариант №11
        \vfill

        \begin{flushright}
            \begin{minipage}{.45\textwidth}
            {
                \hbox{Преподаватель:}
                \hbox{Жданов В. А.}
                \hbox{}
                \hbox{Выполнил:}
                \hbox{Румянцев А. А.}
                \hbox{}
                \hbox{Факультет: СУиР}
                \hbox{Группа: R3341}
                \hbox{Поток: ЭлУСУ R22 бак 1.2}
            }
            \end{minipage}
        \end{flushright}
        \vfill
  
        Санкт-Петербург\\
        2025
        \end{center}
    \end{titlepage}
    
    \tableofcontents

    \newpage
    \section{Цель работы}
    Цель работы -- изучение характеристик операционного усилителя
    (ОУ) в различных режимах работы, исследование ОУ в различных схемах включения.

    
    \section{Таблица 1. Исходные данные}
    Обозначения: $p$ -- коэффициент усиления, $K_1$ и $K_2$ для неинвертирующего сумматора,
    $f_{i,d}$ -- рабочая частота схемы интегратора, дифференциатора
    \begin{center}
        \begin{tabular}{ | m{4em} | m{4em}| m{4em} | m{4em} | m{4em} | } 
        \hline
        ОУ& $p$ &$K_1$ &$K_2$ &$f_{i,d}$, кГц\\ 
        \hline
        LT1037& 8 & 1.5 &3.5 &1\\ 
        \hline
        \end{tabular}
    \end{center}


    \section{Исследование дифференциального усилителя}
    Соберем схему усилителя с дифференциальным входом на ОУ
    для заданного значения коэффициента усиления $p$ в таблице 1.
    В качестве резистора обратной связи используем резистор номиналом 10 кОм.
\end{document}