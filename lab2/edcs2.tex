\documentclass[a4paper, 12pt]{article}
\usepackage[utf8x]{inputenc}
\usepackage[english, russian]{babel}
\usepackage[left=25mm, top=25mm, right=25mm, bottom=25mm]{geometry}
\usepackage{cmap}
\usepackage{indentfirst}
\usepackage{tikz}
\usepackage{float}
\usepackage{amsmath, amsfonts, amssymb}
\usepackage{graphicx}
\usepackage{hyperref}
\usepackage{listings}
\usepackage{caption}
\usepackage{subcaption}
\usepackage{xcolor}
\usepackage{etoolbox}
\usepackage{titlesec}
\usepackage{array}
\pagestyle{plain}
\patchcmd{\tableofcontents}{\contentsname}{\centering\contentsname}{}{}
\titleformat{\section}[block]{\normalfont\large\bfseries\centering}{}{0pt}{}
\titleformat{\subsection}[block]{\normalfont\normalsize\bfseries\centering}{}{0pt}{}
\allowdisplaybreaks
\graphicspath{{src/images/}}
\usetikzlibrary{patterns}
\definecolor{LightGray}{gray}{0.95}
\definecolor{LightGray2}{gray}{0.7}
\hypersetup{
    colorlinks=true,
    linkcolor=blue,
    filecolor=magenta,
    urlcolor=cyan,
    pdftitle={contents setup},
    pdfpagemode=FullScreen,
}


\begin{document}
    \begin{titlepage}

        \begin{center}
        Федеральное государственное автономное образовательное учреждение высшего образования
        «Национальный Исследовательский Университет ИТМО»
        \vfill
        
        \includegraphics[width=0.3\textwidth]{itmo.png} % requires /src/images/itmo.png

        {\large\bf ЛАБОРАТОРНАЯ РАБОТА №2}\\
        {\large\bf ПРЕДМЕТ «ЭЛЕКТРОННЫЕ УСТРОЙСТВА СИСТЕМ УПРАВЛЕНИЯ»}\\
        {\large\bf ТЕМА «СТАБИЛИЗАТОРЫ НАПРЯЖЕНИЯ»}\\
        Вариант №5
        \vfill

        \begin{flushright}
            \begin{minipage}{.45\textwidth}
            {
                \hbox{Преподаватель:}
                \hbox{Жданов В. А.}
                \hbox{}
                \hbox{Выполнил:}
                \hbox{Румянцев А. А.}
                \hbox{}
                \hbox{Факультет: СУиР}
                \hbox{Группа: R3341}
                \hbox{Поток: ЭлУСУ R22 бак 1.2}
            }
            \end{minipage}
        \end{flushright}
        \vfill
  
        Санкт-Петербург\\
        2025
        \end{center}
    \end{titlepage}
    
    \tableofcontents

    \newpage
    \section{Цель работы}
    Цель работы -- исследование и сравнение характеристик различных схемных решений
    стабилизаторов на дискретных элементах и стабилизатора в интегральном исполнении.


    \section{Исходные данные}
    В таблице ниже представлены исходные данные для варианта №5
    \begin{center}
        \begin{tabular}{ | m{4em} | m{2em}| } 
        \hline
        $U_{\text{вых}}$, В& $8$\\ 
        \hline
        $R_{\text{н}}$, Ом& $3500$\\ 
        \hline
        $U_{\text{вх}}$, В& $16$\\ 
        \hline
        \end{tabular}
    \end{center}


    \section{Исследование параметрического стабилизатора}
    \subsection{Выбор стабилитрона}
    Выходное напряжение (напряжение стабилизации) составляет 8 В, тогда возьмем стабилитрон типа EDZV8.2B $\Rightarrow U_{\text{ст}}=8.2$ В.
    При подаче 8.2 В он начнет проводить ток (при $<8.2$ В ничего не будет делать, при $>8.2$ В <<сбросит>>
    лишнее напряжение через себя, удерживая на нагрузке примерно 8.2 В; теперь $U_{\text{вых}}=8.2$ В). Этот стабилитрон имеет рассеиваемую
    мощность $P_\text{ст}=0.15$ Вт, дифференциальное сопротивление $r_{\text{ст}}=30$ Ом
    
    
    \subsection{Расчет параметров схемы}
    Рассчитаем максимальный ток, текущий
    через стабилитрон
    $$
    I_{\text{ст макс}}=\dfrac{P_{\text{ст}}}{U_{\text{ст}}}=\dfrac{0.15}{8.2}=0.0182926829\text{ А}
    $$
    Рассчитаем ток нагрузки
    $$
    I_{\text{н}}=I_{\text{ст}}=\dfrac{U_{\text{вых}}}{R_{\text{н}}}=\dfrac{8.2}{3500}=0.0023428571\text{ А}
    $$
    Рассчитаем номинальное значение тока на стабилитроне
    $$
    I_{\text{ст ном}}=\dfrac{I_{\text{ст макс}}-I_{\text{ст}}}{2}=\dfrac{0.018-0.002}{2}=0.0079749129\text{ А}
    $$
    Определим балластное сопротивление резистора
    $$
    R_{\text{б}}=\dfrac{U_{\text{вх}}-U_{\text{вых}}}{I_{\text{ст ном}}+I_{\text{н}}}=\dfrac{16-8.2}{0.008+0.002}=755.9773090503\text{ Ом}
    $$


    \subsection{Коэффициент стабилизации}
    Определим коэффициент стабилизации
    $$
    k_{\text{ст}}=\left( 1-\dfrac{R_{\text{б}}\left( I_\text{ст ном}+I_{\text{н}} \right)}{U_{\text{вх}}} \right)\cdot\dfrac{R_{\text{б}}+r_{\text{ст}}}{r_{\text{ст}}},
    $$
    $$
    k_{\text{ст}}=\left( 1-\dfrac{755.977\left( 0.008+0.002 \right)}{16} \right)\cdot\dfrac{755.977+30}{30}=13.4271123629;
    $$
    Посчитаем оценку $k_{\text{ст}}$ (приближенно коэффициент стабилизации)
    $$
    \hat{k}_{\text{ст}}=\dfrac{R_{\text{б}}U_{\text{вых}}}{r_{\text{ст}}U_{\text{вх}}}=12.9146123629
    $$


    \subsection{Коэффициент полезного действия}
    Определим коэффициент полезного действия
    $$
    \eta=\dfrac{I_\text{ст ном}U_{\text{ст}}}{U_{\text{вх}}\left( I_{\text{ст ном}}+I_{\text{н}} \right)}=\dfrac{0.008\cdot8.2}{16\left( 0.008+0.002 \right)}=0.3961265720\approx40\%
    $$


    \subsection{Схема параметрического стабилизатора постоянного напряжения}
    Соберем схему параметрического стабилизатора постоянного напряжения с учетом наших расчетов.
    Конденсатор в расчетах не участвовал (со временем перестанет проводить ток)
    -- он нужен для сглаживания пульсаций (фильтр шумов)
    \begin{figure}[H]
        \centering
        \includegraphics[scale=0.22]{1task_scheme_AC.png}
        \captionsetup{skip=0pt}
        \caption{Схема параметрического стабилизатора постоянного напряжения}
        \label{fig:1task_scheme_AC}
    \end{figure}


    \subsection{Влияние сопротивления нагрузки на работу стабилизатора}
    Проверим выходное напряжение цепи и ток на стабилизаторе при постоянном входном напряжении
    16 В и различных сопротивлениях нагрузки. V(n001)$\equiv U_{\text{вх}}$, V(n002)$\equiv U_{\text{вых}}$,
    I(D2)$\equiv I_{\text{ст}}$. Результаты представлены на рис.
    \ref{fig:1task_R1k}--\ref{fig:1task_R100k}
    \begin{figure}[H]
        \centering
        \includegraphics[scale=0.46]{1task_R1k.png}
        \captionsetup{skip=0pt}
        \caption{Выходное напряжение при $R_{\text{н}}=1000$ Ом; $U_{\text{вых ср}}=8.1884$ В}
        \label{fig:1task_R1k}
    \end{figure}
    \begin{figure}[H]
        \centering
        \includegraphics[scale=0.46]{1task_R3_5k.png}
        \captionsetup{skip=0pt}
        \caption{Выходное напряжение при $R_{\text{н}}=3500$ Ом; $U_{\text{вых ср}}=8.1933$ В}
        \label{fig:1task_R3_5k}
    \end{figure}
    \begin{figure}[H]
        \centering
        \includegraphics[scale=0.46]{1task_R10k.png}
        \captionsetup{skip=0pt}
        \caption{Выходное напряжение при $R_{\text{н}}=10000$ Ом; $U_{\text{вых ср}}=8.1941$ В}
        \label{fig:1task_R10k}
    \end{figure}
    \begin{figure}[H]
        \centering
        \includegraphics[scale=0.46]{1task_R100k.png}
        \captionsetup{skip=0pt}
        \caption{Выходное напряжение при $R_{\text{н}}=100000$ Ом; $U_{\text{вых ср}}=8.1945$ В}
        \label{fig:1task_R100k}
    \end{figure}
    \noindent Выходное напряжение с увеличением сопротивления нагрузки немного увеличивается,
    при этом стабилитрон потребляет больше тока. Максимальное значение тока на стабилитроне в 18 мА
    не было достигнуто (при $R_{\text{н}}=100000$ Ом получили $I_{\text{ст}}\approx10.243$ мА).


    \subsection{Скачкообразное изменение нагрузки}
    Подадим скачкообразную нагрузку PULSE(16 18 5m 1u 1u 10m 10m). Входное напряжение
    представлено на рис. \ref{fig:1task_rect_input0}
    \begin{figure}[H]
        \centering
        \includegraphics[scale=0.46]{1task_rect_input0.png}
        \captionsetup{skip=0pt}
        \caption{Скачкообразная нагрузка с 16 В до 18 В}
        \label{fig:1task_rect_input0}
    \end{figure}
    При таком входном напряжении на выходе получаем
    \begin{figure}[H]
        \centering
        \includegraphics[scale=0.46]{1task_rect.png}
        \captionsetup{skip=0pt}
        \caption{Выходное напряжение при скачкообразной нагрузке}
        \label{fig:1task_rect}
    \end{figure}
    \noindent Скачок напряжения на выходе значительно меньше скачка на входе. Стабилизатор
    удержал напряжение в районе 8.2 В. Напряжение до скачка 8.193294 В, после 8.194724 В.


    \subsection{Нагрузки разного вида при скачкообразном изменении входного напряжения}
    Снимем осциллограммы выходных напряжений стабилизатора при скачкообразном
    изменении входного напряжения для нагрузок разного вида. На схеме на рис. \ref{fig:1task_scheme_AC}
    представлена активно-емкостная нагрузка. Для начала построим схему только лишь \textbf{активной} нагрузки
    \begin{figure}[H]
        \centering
        \includegraphics[scale=0.22]{1task_scheme_A.png}
        \captionsetup{skip=0pt}
        \caption{Схема параметрического стабилизатора: активная нагрузка}
        \label{fig:1task_scheme_A}
    \end{figure}
    \noindent Подадим на вход скачкообразный сигнал PULSE(16 18 5m 1u 1u 1m 10m), который
    представлен на рис. \ref{fig:1task_rect_input}
    \begin{figure}[H]
        \centering
        \includegraphics[scale=0.46]{1task_rect_input.png}
        \captionsetup{skip=0pt}
        \caption{Повторяющаяся скачкообразная нагрузка с 16 В до 18 В}
        \label{fig:1task_rect_input}
    \end{figure}
    Посмотрим выходное напряжение при \textbf{активной} скачкообразной нагрузке
    \begin{figure}[H]
        \centering
        \includegraphics[scale=0.46]{1task_rect_A.png}
        \captionsetup{skip=0pt}
        \caption{Выходное напряжение при активной скачкообразной нагрузке}
        \label{fig:1task_rect_A}
    \end{figure}
    \noindent Посмотрим выходное напряжение при \textbf{активно-емкостной} нагрузке. Схема
    была представлена на рис. \ref{fig:1task_scheme_AC}
    \begin{figure}[H]
        \centering
        \includegraphics[scale=0.46]{1task_rect_AC.png}
        \captionsetup{skip=0pt}
        \caption{Выходное напряжение при активно-емкостной скачкообразной нагрузке}
        \label{fig:1task_rect_AC}
    \end{figure}
    \noindent Построим схему для проверки \textbf{активно-индуктивной} нагрузки. Зададим
    значение индуктивности в 1 Гн
    \begin{figure}[H]
        \centering
        \includegraphics[scale=0.22]{1task_scheme_AL.png}
        \captionsetup{skip=0pt}
        \caption{Схема параметрического стабилизатора: активно-индуктивная нагрузка}
        \label{fig:1task_scheme_AL}
    \end{figure}
    \noindent Посмотрим выходное напряжение при \textbf{активно-индуктивной} нагрузке
    \begin{figure}[H]
        \centering
        \includegraphics[scale=0.46]{1task_rect_AL.png}
        \captionsetup{skip=0pt}
        \caption{Выходное напряжение при активно-индуктивной скачкообразной нагрузке}
        \label{fig:1task_rect_AL}
    \end{figure}
    \noindent Построим схему для проверки \textbf{активно-индуктивно-емкостной} нагрузки. Зададим
    значение индуктивности в 1 Гн
    \begin{figure}[H]
        \centering
        \includegraphics[scale=0.22]{1task_scheme_ALC.png}
        \captionsetup{skip=0pt}
        \caption{Схема параметрического стабилизатора: активно-индуктивно-емкостная нагрузка}
        \label{fig:1task_scheme_ALC}
    \end{figure}
    \noindent Посмотрим выходное напряжение при \textbf{активно-индуктивно-емкостной} нагрузке
    \begin{figure}[H]
        \centering
        \includegraphics[scale=0.46]{1task_rect_ALC.png}
        \captionsetup{skip=0pt}
        \caption{Выходное напряжение при активно-индуктивно-емкостной скачкообразной нагрузке}
        \label{fig:1task_rect_ALC}
    \end{figure}
    \noindent Результат лучше всего получился на рис. \ref{fig:1task_rect_ALC}. При увеличении
    емкости конденсатора пульсации будут сглаживаться еще больше.


    \section{Исследование однотранзисторного последовательного линейного стабилизатора}
    \subsection{Выбор стабилитрона}
    Определимся со стабилизатором
    $$
    U_{\text{ст}}=U_{\text{вых}}+0.6=8+0.6=8.6\text{ В}
    $$
    Самые близкие доступные стабилизаторы -- EDZV8.2B на 8.2 В и EDZV9.1B на 9.1 В.
    Сравним по разнице между возможным и желаемым напряжениями на стабилизаторе и возьмем
    напряжение $U_{\text{ст}}$, при котором разница наименьшая
    $$
    9.1-8.6=0.5,\ 8.2-8.6=-0.4,
    $$
    $$
    |-0.4|<|0.5|\Rightarrow\text{ берем EDZV8.2B}
    $$
    Пересчитаем выходное напряжение
    $$
    U_{\text{вых}}=U_{\text{ст}}-0.6=8.2-0.6=7.6\text{ В}
    $$
    В теории теряем 5\% от желаемых 8 В.
    
    
    \subsection{Расчет параметров схемы}
    Далее рассчитаем сопротивление на резисторе.
    Для транзистора 2N3055 выберем коэффициент передачи тока базы $h_{\text{FE мин}}$
    $$\,20\leq h_{\text{FE}}\leq70\Rightarrow h_{\text{FE мин}}=20$$
    Определим минимальное входное напряжение
    $$
    U_{\text{вх мин}}>U_{\text{вых}}+2.5=7.6+2.5=10.1\Rightarrow U_{\text{вх мин}}=11\text{ В},
    $$
    Рассчитаем максимальный выходной ток стабилизатора
    $$
    I_{\text{вых макс}}=h_{\text{FE}}\cdot I_{\text{б}},
    $$
    $$
    I_{\text{б макс}}\approx I_{\text{ст макс}}=\dfrac{P_{\text{ст}}}{U_{\text{ст}}}=\dfrac{0.15}{7.6}=0.0197368421\text{ А},
    $$
    $$
    I_{\text{вых макс}}=20\cdot0.02=0.394736842\text{ А}
    $$
    Теперь посчитаем $R$
    $$
    R\approx\dfrac{U_{\text{вх мин}}h_{\text{FE мин}}}{1.2I_{\text{вых макс}}}=\dfrac{11\cdot20}{1.2\cdot0.395}=464.4444445683\text{ Ом}
    $$


    \subsection{Коэффициент стабилизации}
    Определим коэффициент стабилизации по формуле
    $$
    k_{\text{ст}}=\dfrac{\Delta U_{\text{вх}}}{U_{\text{вх}}}\div\dfrac{\Delta U_{\text{вых}}}{U_{\text{вых}}}\bigg|_{R_\text{н}\text{=const}}
    $$
    Значения $\Delta U_{\text{вых}}$ возьмем с моделирования схемы, представленной на рисунке \ref{fig:2task_scheme_AC}, в LTspice
    при $U_{\text{вх 1}}=16$ В, $U_{\text{вх 2}}=17$ В
    $$
    k_{\text{ст}}=\dfrac{17-16}{16}\div\dfrac{7.9021-7.9013}{7.6}=593.7499999994
    $$


    \subsection{Схема однотранзисторного последовательного линейного стабилизатора постоянного напряжения}
    Построим схему однотранзисторного последовательного линейного стабилизатора постоянного напряжения, учитывая проведенные ранее расчеты
    \begin{figure}[H]
        \centering
        \includegraphics[scale=0.22]{2task_scheme_AC.png}
        \captionsetup{skip=0pt}
        \caption{Схема однотранзисторного последовательного линейного стабилизатора постоянного напряжения}
        \label{fig:2task_scheme_AC}
    \end{figure}


    \subsection{Влияние сопротивления нагрузки на работу стабилизатора}
    Проверим выходное напряжение цепи и ток на стабилизаторе при постоянном
    входном напряжении 16 В и различных сопротивлениях нагрузки. V(n001)$\equiv U_{\text{вх}}$,
    V(n002)$\equiv U_{\text{вых}}$, I(D2)$\equiv I_{\text{ст}}$. Результаты представлены на рис.
    \ref{fig:2task_R1k}--\ref{fig:2task_R100k}
    \begin{figure}[H]
        \centering
        \includegraphics[scale=0.46]{2task_R1k.png}
        \captionsetup{skip=0pt}
        \caption{Выходное напряжение при $R_{\text{н}}=1000$ Ом; $U_{\text{вых ср}}=7.8687$ В}
        \label{fig:2task_R1k}
    \end{figure}
    \begin{figure}[H]
        \centering
        \includegraphics[scale=0.46]{2task_R3_5k.png}
        \captionsetup{skip=0pt}
        \caption{Выходное напряжение при $R_{\text{н}}=3500$ Ом; $U_{\text{вых ср}}=7.9013$ В}
        \label{fig:2task_R3_5k}
    \end{figure}
    \begin{figure}[H]
        \centering
        \includegraphics[scale=0.46]{2task_R10k.png}
        \captionsetup{skip=0pt}
        \caption{Выходное напряжение при $R_{\text{н}}=10000$ Ом; $U_{\text{вых ср}}=7.9284$ В}
        \label{fig:2task_R10k}
    \end{figure}
    \begin{figure}[H]
        \centering
        \includegraphics[scale=0.46]{2task_R100k.png}
        \captionsetup{skip=0pt}
        \caption{Выходное напряжение при $R_{\text{н}}=100000$ Ом; $U_{\text{вых ср}}=7.9878$ В}
        \label{fig:2task_R100k}
    \end{figure}
    \noindent Выходное напряжение с увеличением сопротивления нагрузки немного увеличивается,
    при этом стабилитрон потребляет немного больше тока (в сравнении с результатами для первого задания, представленными
    на рис. \ref{fig:1task_R1k}--\ref{fig:1task_R100k}, увеличение потребления тока значительно меньше).


    \subsection{Скачкообразное изменение нагрузки}
    Выполним моделирование скачкообразного изменения нагрузки аналогично первому заданию
    (входное напряжение представлено на рис. \ref{fig:1task_rect_input0})
    \begin{figure}[H]
        \centering
        \includegraphics[scale=0.46]{2task_rect.png}
        \captionsetup{skip=0pt}
        \caption{Выходное напряжение при скачкообразной нагрузке}
        \label{fig:2task_rect}
    \end{figure}
    \noindent Скачок напряжения на выходе значительно меньше скачка на входе. Стабилизатор
    удержал напряжение в районе 8 В. Напряжение до скачка 7.9877738 В, после 7.989481 В.


    \subsection{Нагрузки разного вида при скачкообразном изменении входного напряжения}
    Снимем осциллограммы выходных напряжений стабилизатора при скачкообразном
    изменении входного напряжения для нагрузок разного вида. На схеме на рис. \ref{fig:2task_scheme_AC}
    представлена активно-емкостная нагрузка. Для начала построим схему только лишь \textbf{активной} нагрузки
    \begin{figure}[H]
        \centering
        \includegraphics[scale=0.22]{2task_scheme_A.png}
        \captionsetup{skip=0pt}
        \caption{Схема параметрического стабилизатора: активная нагрузка}
        \label{fig:2task_scheme_A}
    \end{figure}
    \noindent Подадим на вход скачкообразный сигнал аналогично первому заданию (см рис. \ref{fig:1task_rect_input}).
    Посмотрим выходное напряжение при \textbf{активной} скачкообразной нагрузке
    \begin{figure}[H]
        \centering
        \includegraphics[scale=0.46]{2task_rect_A.png}
        \captionsetup{skip=0pt}
        \caption{Выходное напряжение при активной скачкообразной нагрузке}
        \label{fig:2task_rect_A}
    \end{figure}
    \noindent Посмотрим выходное напряжение при \textbf{активно-емкостной} нагрузке. Схема
    была представлена на рис. \ref{fig:2task_scheme_AC}
    \begin{figure}[H]
        \centering
        \includegraphics[scale=0.46]{2task_rect_AC.png}
        \captionsetup{skip=0pt}
        \caption{Выходное напряжение при активно-емкостной скачкообразной нагрузке}
        \label{fig:2task_rect_AC}
    \end{figure}
    \noindent Построим схему для проверки \textbf{активно-индуктивной} нагрузки. Зададим
    значение индуктивности в 100 Гн
    \begin{figure}[H]
        \centering
        \includegraphics[scale=0.22]{2task_scheme_AL.png}
        \captionsetup{skip=0pt}
        \caption{Схема параметрического стабилизатора: активно-индуктивная нагрузка}
        \label{fig:2task_scheme_AL}
    \end{figure}
    \noindent Посмотрим выходное напряжение при \textbf{активно-индуктивной} нагрузке
    \begin{figure}[H]
        \centering
        \includegraphics[scale=0.46]{2task_rect_AL.png}
        \captionsetup{skip=0pt}
        \caption{Выходное напряжение при активно-индуктивной скачкообразной нагрузке}
        \label{fig:2task_rect_AL}
    \end{figure}
    \noindent Построим схему для проверки \textbf{активно-индуктивно-емкостной} нагрузки. Зададим
    значение индуктивности в 100 Гн
    \begin{figure}[H]
        \centering
        \includegraphics[scale=0.22]{2task_scheme_ALC.png}
        \captionsetup{skip=0pt}
        \caption{Схема параметрического стабилизатора: активно-индуктивно-емкостная нагрузка}
        \label{fig:2task_scheme_ALC}
    \end{figure}
    \noindent Посмотрим выходное напряжение при \textbf{активно-индуктивно-емкостной} нагрузке
    \begin{figure}[H]
        \centering
        \includegraphics[scale=0.46]{2task_rect_ALC.png}
        \captionsetup{skip=0pt}
        \caption{Выходное напряжение при активно-индуктивно-емкостной скачкообразной нагрузке}
        \label{fig:2task_rect_ALC}
    \end{figure}
    \noindent Результат лучше всего получился на рис. \ref{fig:2task_rect_ALC}.
    При увеличении емкости конденсатора пульсации будут сглаживаться еще больше.


    \section{Исследование компенсационного стабилизатора}
    \subsection{Выбор стабилитрона}
    Выберем стабилитрон измерительного моста со значением напряжения
    стабилизации, равным половине выходного напряжения стабилизатора
    $$
    U_{\text{ст 1}}=\dfrac{U_{\text{вых}}}{2}=\dfrac{8}{2}=4\text{ В}
    $$
    При этом условии обеспечивается наилучшая стабилизация.
    Выбираем стабилитрон EDZV3.9B -- его напряжение стабилизации
    близко к расчитанному. Тогда
    $$
    U_{\text{ст 1}}=3.9\text{ В},\ I_{\text{ст 1}}=5\text{ мА}
    $$


    \subsection{Расчет параметров схемы}
    Определим значение сопротивления балластного резистора
    $R_{\text{б}}$. Падение напряжения на балластном сопротивлении
    составляет
    $$
    U_{R_\text{б}}=U_{\text{вых}}-U_{\text{ст 1}}=8-3.9=4.1\text{ В}
    $$
    Значение сопротивления балластного резистора может быть рассчитано по закону Ома
    $$
    R_{\text{б}}=\dfrac{U_{R_\text{б}}}{I_{\text{ст 1}}}=\dfrac{4.1}{5\cdot10^{-3}}=820\text{ Ом}
    $$


    Рассчитаем значения делителя напряжения $R_1$...$R_3$.
    Традиционно переменный резистор $R_2$ используется для
    возможности подстройки выходного напряжения схемы.
    В LTspice есть проблема -- отсутствие в пакете переменного
    резистора, поэтому в работе будем использовать делитель на
    базе двух сопротивлений $R_1$ и $R_3$.
    Зададим ток через делитель в 10 раз меньше,
    чем ток стабилизации стабилитрона
    $$
    I_{\text{дел}}=\dfrac{I_{\text{ст 1}}}{10}=\dfrac{5\cdot10^{-3}}{10}=0.5 \text{ мА} 
    $$
    В стабилизаторе компенсационного типа транзисторы работают
    в активном режиме. Известно, что в активном режиме напряжение
    между базой и эмиттером биполярного транзистора составляет
    $0.6$...$0.7$ В, выберем значение
    $$
    U_{\text{БЭ}}=0.65 \text{ В}
    $$
    Таким образом потенциал базы,
    равный падению напряжения на
    резисторе $R_3$ составляет
    $$
    U_{R_3}=U_{\text{ст 1}}+0.65=4.55\text{ В}
    $$
    А на $R_1$
    $$
    U_{R_1}=U_{\text{вых}}-U_{R_3}=8-4.55=3.45\text{ В}
    $$
    Зная падения напряжения на резисторах и ток через делитель,
    можно рассчитать значения сопротивлений по закону Ома
    $$
    R_3=\dfrac{U_{R_3}}{I_{\text{дел}}}=\dfrac{4.55}{0.5\cdot10^{-3}}=9.1\text{ кОм},
    $$
    $$
    R_1=\dfrac{U_{R_1}}{I_{\text{дел}}}=\dfrac{3.45}{0.5\cdot10^{-3}}=6.9\text{ кОм};
    $$
    В реальности мы не сможем найти резисторы
    с такими сопротивлениями, вопрос балансировки
    схемы нам бы помог решить переменный резистор $R_2$.
    Выберем второй источник опорного напряжения,
    в качестве источника опорного напряжения выберем стабилитрон
    EDZV13B с напряжением стабилизации $U_{\text{ст 2}}=13$ В
    и током $I_{\text{ст 2}}=5$ мА.
    Найдем значение сопротивления балластного
    резистора $R_{\text{СМ}}$ для номинальный значений параметров схемы
    $$
    R_{\text{СМ}}=\dfrac{U_{\text{вх}}-U_{\text{ст 2}}}{I_\text{ст 2}}=\dfrac{16-13}{5\cdot10^{-3}}=600\text{ Ом}
    $$


    Рассчитаем значение сопротивления резистора $R_\text{К}$.
    Для стабильной работы цепи опорного напряжения (транзистор 2),
    необходимо, чтобы $R_\text{К}$ не оказывал на эту цепь шунтирующего действия.
    Поэтому ток $R_\text{К}$ должен быть не менее, чем в 2 раза меньше тока
    стабилитрона
    $$
    I_{R_\text{К}}=\dfrac{I_{\text{ст 2}}}{2}=\dfrac{5\cdot10^{-3}}{2}=2.5\text{ мА}
    $$
    Кроме того, на нём падает разность между входным
    и выходным напряжениями
    $$
    U_{R_\text{К}}=U_{\text{вх}}-U_{\text{вых}}=16-8=8\text{ В},
    $$
    $$
    R_\text{К}=\dfrac{U_{R_\text{К}}}{I_{R_\text{К}}}=\dfrac{8}{2.5\cdot10^{-3}}=3.2\text{ кОм};
    $$


    Выберем транзисторы для стабилизатора. Пусть первый транзистор Q1 будет
    маломощный биполярный транзистор с максимальным напряжением между коллектором и эмиттером более 30 В -- выберем 2N3904 с параметрами
    $$
    U_{\text{КЭ}_\text{макс}}=40\text{ В},\ I_\text{К}=200\text{ мА},\ h_{21_\text{Э}}=100...300;
    $$
    Возьмем такой же транзистор в качестве выходного (транзистор 3), так как его ток коллектора больше максимального выходного тока. В качестве второго транзистора
    выберем 2N2222 с параметрами
    $$
    U_{\text{КЭ}_\text{макс}}=40\text{ В},\ I_\text{К}=600\text{ мА},\ h_{21_\text{Э}}=75...300;
    $$
    Рассчитаем ток нагрузки
    $$
    I_\text{н}=\dfrac{8}{3500}=0.0022857143\text{ А}
    $$
    Определим ток базы третьего транзистора
    $$
    I_{\text{Б}_\text{Т3}}=\dfrac{I_\text{н}}{h_{21_\text{Э мин}}}=\dfrac{0.002}{100}=0.0000228571\text{ А}
    $$
    Транзистор усиливает ток, текущий через $R_\text{К}$. Для выбранного транзистора 3 имеем выходной ток
    $$
    I_{\text{К}_\text{Т3}}=I_{\text{R}_\text{К}}h_{21_\text{Э мин}}=2.5\cdot10^{-3}\cdot75=0.1875\text{ А}
    $$
    Рассчитаем $R_\text{Э}$ при $h_{21_\text{Э}}=85$
    $$
    R_\text{Э}=\dfrac{U_\text{БЭ}}{I_\text{н}}h_{21_\text{Э}}=\dfrac{0.65}{0.002}\cdot85=24171.8748489258\text{ Ом}
    $$


    \subsection{Коэффициент стабилизации}
    Рассчитаем коэффициент стабилизации аналогично заданию 2
    $$
    k_{\text{ст}}=\dfrac{\Delta U_{\text{вх}}}{U_{\text{вх}}}\div\dfrac{\Delta U_{\text{вых}}}{U_{\text{вых}}}\bigg|_{R_\text{н}\text{=const}}=\dfrac{17-16}{16}\div\dfrac{8.5379-8.5376}{8}=1666.6666666706
    $$


    \subsection{Схема компенсационного стабилизатора постоянного напряжения на биполярном транзисторе}
    Построим схему с учетом проведенных ранее расчетов
    \begin{figure}[H]
        \centering
        \includegraphics[scale=0.22]{3task_scheme_A.png}
        \captionsetup{skip=0pt}
        \caption{Схема компенсационного стабилизатора постоянного напряжения на биполярном транзисторе}
        \label{fig:3task_scheme_A}
    \end{figure}


    \subsection{Влияние сопротивления нагрузки на работу стабилизатора}
    Проверим выходное напряжение цепи и ток на стабилизаторе при постоянном
    входном напряжении 16 В и различных сопротивлениях нагрузки. V(n001)$\equiv U_{\text{вх}}$,
    V(n002)$\equiv U_{\text{вых}}$, I(D2)$\equiv I_{\text{ст}}$. Результаты представлены на рис.
    \ref{fig:3task_R1k}--\ref{fig:3task_R100k}
    \begin{figure}[H]
        \centering
        \includegraphics[scale=0.46]{3task_R1k.png}
        \captionsetup{skip=0pt}
        \caption{Выходное напряжение при $R_{\text{н}}=1000$ Ом; $U_{\text{вых ср}}=8.5370$ В}
        \label{fig:3task_R1k}
    \end{figure}
    \begin{figure}[H]
        \centering
        \includegraphics[scale=0.46]{3task_R3_5k.png}
        \captionsetup{skip=0pt}
        \caption{Выходное напряжение при $R_{\text{н}}=3500$ Ом; $U_{\text{вых ср}}=8.5376$ В}
        \label{fig:3task_R3_5k}
    \end{figure}
    \begin{figure}[H]
        \centering
        \includegraphics[scale=0.46]{3task_R10k.png}
        \captionsetup{skip=0pt}
        \caption{Выходное напряжение при $R_{\text{н}}=10000$ Ом; $U_{\text{вых ср}}=8.5378$ В}
        \label{fig:3task_R10k}
    \end{figure}
    \begin{figure}[H]
        \centering
        \includegraphics[scale=0.46]{3task_R100k.png}
        \captionsetup{skip=0pt}
        \caption{Выходное напряжение при $R_{\text{н}}=100000$ Ом; $U_{\text{вых ср}}=8.5379$ В}
        \label{fig:3task_R100k}
    \end{figure}
    \noindent Выходное напряжение с увеличением сопротивления нагрузки почти не изменяется,
    потребление стабилитроном тока тоже.


    \subsection{Скачкообразное изменение нагрузки}
    Выполним моделирование скачкообразного изменения нагрузки аналогично первому заданию
    (входное напряжение представлено на рис. \ref{fig:1task_rect_input0})
    \begin{figure}[H]
        \centering
        \includegraphics[scale=0.46]{3task_rect.png}
        \captionsetup{skip=0pt}
        \caption{Выходное напряжение при скачкообразной нагрузке}
        \label{fig:3task_rect}
    \end{figure}
    \noindent Скачок напряжения на выходе значительно меньше скачка на входе. Стабилизатор
    удержал напряжение в районе 8.5 В. Напряжение до скачка 8.537611 В, после 8.538035 В.


    \subsection{Нагрузки разного вида при скачкообразном изменении входного напряжения}
    Снимем осциллограммы выходных напряжений стабилизатора при скачкообразном
    изменении входного напряжения для нагрузок разного вида аналогично первому заданию (см рис. \ref{fig:1task_rect_input}).
    Проверим \textbf{активную} нагрузку (схема на рис. \ref{fig:3task_scheme_A})
    \begin{figure}[H]
        \centering
        \includegraphics[scale=0.46]{3task_rect_A.png}
        \captionsetup{skip=0pt}
        \caption{Выходное напряжение при активной скачкообразной нагрузке}
        \label{fig:3task_rect_A}
    \end{figure}
    \noindent Проверим \textbf{активно-емкостную} нагрузку. Добавим в схему конденсатор
    \begin{figure}[H]
        \centering
        \includegraphics[scale=0.22]{3task_scheme_AC.png}
        \captionsetup{skip=0pt}
        \caption{Схема компенсационного стабилизатора: актино-емкостная нагрузка}
        \label{fig:3task_scheme_AC}
    \end{figure}
    \begin{figure}[H]
        \centering
        \includegraphics[scale=0.46]{3task_rect_AC.png}
        \captionsetup{skip=0pt}
        \caption{Выходное напряжение при активно-емкостной скачкообразной нагрузке}
        \label{fig:3task_rect_AC}
    \end{figure}
    \noindent Построим схему для проверки \textbf{активно-индуктивной} нагрузки. Зададим
    значение индуктивности в 1 Гн
    \begin{figure}[H]
        \centering
        \includegraphics[scale=0.22]{3task_scheme_AL.png}
        \captionsetup{skip=0pt}
        \caption{Схема компенсационного стабилизатора: активно-индуктивная нагрузка}
        \label{fig:3task_scheme_AL}
    \end{figure}
    \noindent Посмотрим выходное напряжение при \textbf{активно-индуктивной} нагрузке
    \begin{figure}[H]
        \centering
        \includegraphics[scale=0.46]{3task_rect_AL.png}
        \captionsetup{skip=0pt}
        \caption{Выходное напряжение при активно-индуктивной скачкообразной нагрузке}
        \label{fig:3task_rect_AL}
    \end{figure}
    \noindent Построим схему для проверки \textbf{активно-индуктивно-емкостной} нагрузки. Зададим
    значение индуктивности в 1 Гн, емкости в 1 мкФ
    \begin{figure}[H]
        \centering
        \includegraphics[scale=0.22]{3task_scheme_ALC.png}
        \captionsetup{skip=0pt}
        \caption{Схема компенсационного стабилизатора: активно-индуктивно-емкостная нагрузка}
        \label{fig:3task_scheme_ALC}
    \end{figure}
    \noindent Посмотрим выходное напряжение при \textbf{активно-индуктивно-емкостной} нагрузке
    \begin{figure}[H]
        \centering
        \includegraphics[scale=0.46]{3task_rect_ALC.png}
        \captionsetup{skip=0pt}
        \caption{Выходное напряжение при активно-индуктивно-емкостной скачкообразной нагрузке}
        \label{fig:3task_rect_ALC}
    \end{figure}
    \noindent Для хорошего результата достаточно добавить катушку индуктивности со значением 1 Гн перед резистором $R_\text{н}$
    (см. рис. \ref{fig:3task_scheme_AL}, \ref{fig:3task_rect_AL}). В целом результат на рис. \ref{fig:3task_rect_ALC} получился лучше всего.


    \section{Вывод}
    В ходе работы были рассмотрены различные схемы со стабилизаторами напряжения:
    параметрический стабилизатор, однотранзисторный последовательный
    линейный стабилизатор, компенсационный стабилизатор. Для каждого
    случая были проведены расчеты компонентов схемы,
    коэффициента стабилизации. Было исследовано влияние скачкообразной
    нагрузки на работу схем.
\end{document}